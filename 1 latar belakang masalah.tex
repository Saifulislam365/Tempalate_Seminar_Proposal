\section{Latar Belakang Masalah}

Pendidikan adalah salah satu faktor penting untuk kemajuan dan kemandirian suatu bangsa. Semakin maju pendidikan bangsa, maka akan semakin maju dan mandiri bangsa tersebut serta dapat cepat berkembang. Dengan pendidikan para generasi muda mengalami proses perkembangan untuk meningkatkan kualitasnya. Pendidikan nasional berfungsi mengembangkan kemampuan dan membentuk watak serta peradaban bangsa yang bermartabat dalam rangka mencerdaskan kehidupan bangsa, bertujuan untuk mengem-bangkan potensi peserta didik agar menjadi manusia yang beriman dan bertakwa kepada Tuhan Yang Maha Esa, berakhlaq mulia, sehat, berilmu, cakap, kreatif, mandiri dan menjadi warga negara yang demokratis serta bertanggungjawab (Depdiknas, 2003). Berdasarkan tujuan pendidikan tersebut maka kualitas dan manajemen pembelajaran di bebagai lembaga pendidikan perlu ditingkatkan. Salah satu indikator kualitas dan manajemen sekolah atau lembaga pendidikan dapat dilihat dari minat belajar siswa.

Dalam artikelnya (Purnawi, 2019) mengemukakan minat adalah sebuah kecenderungan yang dalam  pelaksanaannya  dilakukan  secara  menetap  dengan  tujuan  untuk  memperhatikan  dan mengenang  beberapa  aktivitas.  Sedangkan  dalam artikel lain (Sabri, Ahmad, 2005) minat  diartikan  sebagai kecenderungan untuk selalu memperhatikan dan mengingat suatu hal secara terus menerus, minat memiliki kaitan erat dengan perasaan senang, karena itu dapat dikatakan minat itu terjadi karena sikap senang kepada sesuatu, orang berminat pada sesuatu berarti ia sikapnya senang kepada sesuatu. Lalu menurut (Muhibbin, 2006) minat adalah sebuah kecenderungan dan kegairahan yang tinggi atau keinginan yang besar terhadap sesuatu. 




Berdasarkan pendapat di atas maka dapat disimpulkan minat merupakan suatu perasaan suka atau ketertarikan yang kuat dan bersumber dari dalam diri seseorang untuk terus semangat dan aktif tanpa adanya paksaan. Maka dari itu minat belajar merupakan keinginan kuat untuk terus semangat mengerjakan dan aktif dalam kegiatan pembelajaran. Minat belajar sangat besar pengaruhnya dalam kegiatan pembelajaran berlangsung.
Seorang pengajar terutama seorang guru dituntut untuk bisa memahami seorang siswa, apakah seorang siswa tersebut mempunyai minat belajar atau minat belajarnya sudah tidak ada. Ketiadaan minat belajar dalam diri siswa adalah hal yang harus dicari penyebabnya sehingga dapat diatasi secara dini. Banyak sekali faktor yang mempenagruhi minat belajar. Misalnya, lingkungan serta  keluarga yang kami kemas dalam label jumlah keluarga dan partisipasi siwa, sumber media dan fasilitas (Putrina Mesra, Eko Kuntarto, Faizal Chan, 2021), jumlah jam belajar atau lama belajar  (Lestari, 2015). Sehingga seorang guru harus berupaya keras untuk menentukan minat belajar siswanya. Sehingga untuk melakukan pendeteksian terhadapap minat belajar ini sangat penting.
Dalam melakukan sebuah prediksi sebuah kasus, ada sebuah metode  dengan menggunakan sebuah mechine learning. Machine learning adalah ilmu yang mempelajari tentang algoritma komputer yang bisa mengenali pola-pola didalam data, dengan tujuan untuk mengubah dan menjadikan sebuah tindakan nyata dengan sedikit mungkin campur tangan manusia (Dios Kurniawan, 2020). 
 Dalam machine learning banyak sekali algoritma yang bisa digunakan.  Salah satunya adalah model Churn Prediction. Churn Prediction merupakan model prediktif yang biasa digunakan oleh perusahaan perusahan industri untuk melakukan prediksi terhadap pelangganya yang mana penerapannya sebagai upaya untuk pencegahan sebuah churn/berhenti berlangganan (Dios Kurniawan, 2020). Oleh sebab itu dalam penelitian ini, peneliti ingin menggunakan model yang sama untuk memprediksi minat belajar , Penulis memberikan sebuah solusi yang dikemas dalam sebuah algoritma machine learning yang bisa memprediksi siswa yang masih memiliki minat atau sudah kehilangan minat belajarnya sehingga guru dapat memberikan pencegahan dini terhadapap siswa yang kurang berminat belajar.