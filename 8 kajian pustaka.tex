\section{Kajian Pustaka}

Banyak faktor yang mempenagruhi minat belajar. Misalnya, lingkungan serta  keluarga yang kami kemas dalam label jumlah keluarga dan partisipasi siwa, sumber media dan fasilitas (Putrina Mesra, Eko Kuntarto, Faizal Chan, 2021) \citep{sugiyono2013metode}, jumlah jam belajar atau lama belajar  (Lestari, 2015). Pada penelitain ini peneliti menggunakan faktor faktor tersebut untuk memprediksi dengan algoritma machine learning. Machine learning adalah ilmu yang mempelajari tentang algoritma komputer yang bisa mengenali pola-pola didalam data, dengan tujuan untuk mengubah dan menjadikan sebuah tindakan nyata dengan sedikit mungkin campur tangan manusia (Dios Kurniawan, 2020). Dalam machine learning banyak sekali algoritma yang bisa digunakan.  Salah satunya adalah model Churn Prediction. Churn Prediction merupakan model prediktif yang biasa digunakan oleh perusahaan perusahan industri untuk melakukan prediksi terhadap pelangganya yang mana penerapannya sebagai upaya untuk pencegahan sebuah churn/berhenti berlangganan (Dios Kurniawan, 2020)\footnote{Dios Kurniawan 2020}\citep{machinelearning}
