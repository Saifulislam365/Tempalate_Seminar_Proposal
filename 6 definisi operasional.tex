\section{Definisi Operasional}

Untuk memberi batas ruang lingkup yang hendak diteliti maka peneliti membuat definisi operasional sebagai berikut: 
\begin{enumerate}
\item Churn Prediction Churn Prediction merupakan model prediktif yang biasa digunakan oleh perusahaan perusahan industri untuk melakukan prediksi terhadap pelangganya yang mana penerapannya sebagai upaya untuk pencegahan sebuah churn/berhenti berlangganan (Dios Kurniawan, 2020).

\item Minat belajar merupakan keinginan kuat untuk terus semangat mengerjakan dan aktif dalam kegiatan pembelajaran. Minat belajar         sangat besar pengaruhnya dalam kegiatan pembelajaran berlangsung.

\item Faktor Faktor minat belajar dalam beberapa artikel banyak hal yang mempengaruhinya misalnya lingkungan, keluarga, partisipasi siwa, sumber media dan fasilitas (Putrina Mesra, Eko Kuntarto, Faizal Chan, 2021), jumlah jam belajar atau lama belajar  (Lestari, 2015). 

\item Churn Prediction merupakan model prediksi yang biasa digunakan oleh perusahaan perusahan industri untuk melakukan prediksi terhadap pelangganya yang mana penerapannya sebagai upaya untuk pencegahan sebuah churn/berhenti berlangganan (Dios Kurniawan, 2020)
\end{enumerate}